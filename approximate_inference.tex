\chapter{Approximate inference}

动机,准确推理不可行,转化为用一些可行的分布去近似那些不可
行的分布, 在保证可行的情况下,尽可能提供丰富的分布,从中找出
最优的。
\section{变分推理}
\begin{enumerate}
\item 变化推理的目标,后验分布,似然函数
\begin{equation}
p(x)
\end{equation}
\begin{equation}
p(z|x)
\end{equation}

\item 对数似然等于联合相对熵减去条件相对熵(对数似然分解)
\begin{equation}
\ln p(x) = L(q) + KL(q||p)
\end{equation}
\begin{equation}
L(q) = KL(q||p(x,z))
\end{equation}
\begin{equation}
KL(q||p) = KL(q||p(x|z))
\end{equation}
与以往不同,这里没有出现参数$\theta$,这是因为在这里所有的参数也
看作是随机变量,因此Z已经包含了参数。
\end{enumerate}

\section{因式分布}

\begin{equation}
q(z) = \prod_{m=1}^Mq_m(z_m)
\end{equation}
参照模图型分解
\subsection{最优解}
因式最优解等于非因式联合期望
\begin{equation}
\ln q_j = E_{-j}[\ln p(x, z)] + const
\end{equation}

\section{问题}
\begin{enumerate}
\item 为什么使用近似推理?
\item 什么是变分?
\item 变分的关键地方是什么?
\item EM算法用到了哪些技巧?是否可以重用?
\item 分离技巧,RVM,变分都有体现。
\end{enumerate}

