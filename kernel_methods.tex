\chapter{Kernel Methods}
\section{问题}
\begin{enumerate}
\item 什么是核?
\item 什么是高斯过程?
\item 为什么使用高斯过程?
\item P292 固定核?
\item 对偶形式的表达,出现了参数个数的变化,由$|w|$变为
$|a|$,是否会过拟合,能带来哪些好处,特征多的影响,特征少的
影响。例如文档分类,特征多,训练集少。
\end{enumerate}
\section{核}
\begin{enumerate}
\item 高斯核
\begin{equation}
k(x_m, x_n) = exp(-||x_m - x_n||^2/2\sigma^2)
\end{equation}
\item 指数核
\begin{equation}
k(x_m, x_n) = exp(-\theta|x_m - x_n|)
\end{equation}
\end{enumerate}

\section{对偶表示}
\subsection{线性回归}
\begin{equation}
y(x) = w^T\phi(x)
\end{equation}
\subsection{对偶表示}
\begin{equation}
y(x) = a^T\Phi\phi(x) = a^Tk(x)
\end{equation}
\subsection{二者之间的关系}
\begin{equation}
J(a) = \frac{1}{2}\sum_{n=1}^N(k(x_n)^Ta - t_n)^2
= \frac{1}{2}||\Phi\Phi^Ta - t||^2
\end{equation}

\begin{equation}
J(w) = \frac{1}{2}\sum_{n=1}^N(\phi(x_n)^Tw - t_n)^2
= \frac{1}{2}||\Phi w - t||^2
\end{equation}

\subsection{协方差矩阵}

\begin{equation}
cov(y) = E(yy^T) = E(\Phi ww^T\Phi^T)
= \Phi E(ww^T) \Phi^T
= \Phi cov(w)\Phi^T
\end{equation}

\begin{equation}
C_{N+1} = \begin{pmatrix}
C_N & k\\
k^T & c
\end{pmatrix}
\end{equation}

对偶形式的优点是,不用显示的写出特征空间向量,只关心
$k(x_m, x_n)$, 核函数完成了由特征空间到核空间的转化。
最重要的是,可以使用不同的核函数进行替换。

\section{高斯过程}
\begin{enumerate}
\item 高斯过程是定义在y上联合分布服从高斯分布的概率分布,
简单说就是高斯过程就是
定义在y上的高斯分布。高斯过程可以由完全由二次统计决定,即均值和方差。
根据高斯分布的性质,高斯分布的边缘分布和条件分布都是高斯分布。
y的取值范围是R,因此是一个无穷维。在实际应用中,只考虑训练集大小
的维度。
回归问题的目标是$p(y_{N+1}) = N(y_{N+1}|\mu_{N+1}, \sigma_{N+1})$。
因此用高斯过程解决回归问题的关键是如何求出均值和方差。
高斯过程中的协方差$E[y(x_n)y(x_m)]$可以用核的方式给出来。

定一个实矩阵 A,矩阵 $A^TA$ 是 A 的列向量的格拉姆矩阵,
而矩阵$AA^T$ 是 A 的行向量的格拉姆矩阵。
\item 高斯过程与核函数\\
高斯过程协方差矩阵自然的与核函数联系在一起。

\end{enumerate}
\section{Summary}
\begin{enumerate}
\item 高斯过程与线性回归之间的对偶
\item 高斯过程与隐变量,对于输出y,y和y之间
实际上是有关系的,通过指定条件w(相当于一个隐变量),
从而使得y只依赖于w和x,与产生式模型的比较。
\item 高斯过程与线性模型\\
线性模型用w表示了x和y之间的关系,并且通过w之间的
协方差表示y之间的协方差。高斯过程抛弃了w,直接表示
y之间的关系,这种关系通过核表示出来。高斯过程用核$k(x_m,x_n)$
取代了线性模型中的w。
\item 回归与分类,回归问题+一个激活函数,转变为分类问题
\end{enumerate}


