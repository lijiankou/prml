\chapter{Probability Distribution}
\label{probability_distribution}

\section{高斯分布}
\begin{equation}
N(x|\mu, \Sigma), \Lambda \equiv \Sigma^{-1}
\end{equation}

\begin{equation}
\begin{aligned}
x = \begin{pmatrix}
 x_a\\ 
 x_b
  \end{pmatrix},
\mu = 
\begin{pmatrix}
 \mu_a \\ 
 \mu_b
  \end{pmatrix}
\end{aligned}
\end{equation}

\begin{equation}
\begin{aligned}
\Sigma = \begin{pmatrix}
\Sigma_{aa}&\Sigma_{ab}\\
\Sigma_{ba}&\Sigma_{bb}
\end{pmatrix},
\Lambda = 
\begin{pmatrix}
\Lambda_{aa}& \Lambda_{ab}\\
\Lambda_{ba}& \Lambda_{bb}
\end{pmatrix}
\end{aligned}
\end{equation}

\subsection{条件分布}
\begin{equation}
\begin{aligned}
p(x_a|x_b) = N(x_a|\mu_{a|b}, \Lambda_{aa}^{-1})
\\
\mu_{a|b} = \mu_a - \Lambda_{aa}^{-1}\Lambda_{ab}(x_b - \mu_b)
\end{aligned}
\end{equation}

\subsection{边缘分布}
\begin{equation}
\begin{aligned}
p(x_a) = N(x_a|\mu_a, \Sigma_{aa})
\end{aligned}
\end{equation}

\section{贝叶斯推理}
\begin{equation}
p(x) = N(x|\mu, \Lambda^{-1})
\end{equation}

\begin{equation}
p(y|x) = N(y|Ax + b, L^{-1})
\end{equation}

\begin{equation}
p(y) = N(y|A\mu + b, L^{-1} + A\Lambda^{-1}A^T)
\end{equation}

\begin{equation}
p(x|y) = N(x|\Sigma\{A^TL(y-b) + \Lambda\mu\}, \Sigma)
\end{equation}

\begin{equation}
\Sigma = (\Lambda + A^TLA)^{-1}
\end{equation}

\subsection{伽玛分布}
精度的符号问题,$\alpha$是一种常用记号,在线性回归里面
经常使用$\alpha,\beta$,使用$\lambda$也很容易理解,在线性代数里面
$\lambda$经常用表示特征值,协方差矩阵用SAS分解以后,就是一个对角
阵,每个元素就是特征值,这也就很容易理解为什么精度矩阵用$\Lambda$了,
因为它本就是特征值的矩阵。
伽玛分布常作为高斯分布的共轭出现。

\begin{equation}
Ga(\lambda|a, b) = \frac{b^a}{\Gamma(a)}\lambda^{a-1}e^{-b\lambda}
= z \lambda^{a-1}e^{-b\lambda}
\end{equation}

\begin{equation}
p(x|\lambda) = \prod_{n=1}^NN(x_n|\mu, \lambda^{-1})
= z \lambda^{\frac{N}{2}}e^{-\frac{\lambda}{2}\sum_{n=1}^N(x_n-\mu)^2}
= z \lambda^ke^{-l \lambda}
\end{equation}

\subsection{高斯伽玛分布}
\begin{equation}
p(\mu, \lambda) = p(\mu|\lambda)p(\lambda)
= N(\mu|\mu_0, (\beta \lambda)^{-1})Ga(\lambda|a, b)
\end{equation}

\subsection{指数族}

\begin{enumerate}
\item 指数族
\begin{equation}
p(x|\eta) = h(x)g(\eta)exp\{\eta^Tu(x)\}
\end{equation}
\end{enumerate}

\subsection{极大似然估计和充分统计}

\begin{enumerate}
\item 似然函数
\begin{equation}
p(X|\eta) = (\prod_{n=1}^Nh(x_n))g(\eta)^N \exp \{
\eta^T \sum_{n=1}^N u(x_n)
\}
\end{equation}
\item 充分统计
\begin{equation}
-\nabla\ln(\eta_{ML}) = \frac{1}{N}\sum_{n=1}^N{u(x_n)}
\end{equation}
\end{enumerate}

\subsection{共轭先验}
\begin{equation}
p(\eta|x, v) = f(x, v)g(\eta)^v\exp\{v\eta^Tx\}
\end{equation}


\subsection{问题}
\begin{enumerate}
\item 如何根据2.228的定义求出$\eta_{ML}$?\\
根据公式2.203, 2.217以及2.223可以知道,$g(\eta)$与均值与方差的关系,
因此只要我们知道了均值
与方差就可以知道$\eta_{ML}$。$\ln g(\eta)$称作配分函数,
配分函数的一个很好性质就是它的导数就$E(u(x))$,二次导
数是$var(u(x))$,因此由2.228可以得到$\eta$。
\end{enumerate}
