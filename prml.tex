\documentclass[graybox, envcountchap]{styles/svmult}
\usepackage{CJKutf8}
\author{lijiankoucoco@163.com}
\title{Machine Learning}
\usepackage{amssymb,amsmath,bm}
\DeclareMathAlphabet{\mathcal}{OMS}{cmsy}{m}{n}
\usepackage{textcomp}
\newcommand\abs[1]{\left\lvert#1\right\rvert}
\usepackage{longtable}
\usepackage{ulem}
\usepackage{algorithm2e}
\usepackage{tocbibind}
\renewcommand{\bibname}{References}
%\usepackage{mathptmx}        % selects Times Roman as basic font
\usepackage{helvet}          % selects Helvetica as sans-serif font
\usepackage{courier}         % selects Courier as typewriter font
\usepackage{makeidx}         % allows index generation
\usepackage{graphicx}        % standard LaTeX graphics tool
\usepackage{multicol}        % used for the two-column index
\usepackage[bottom]{footmisc}% places footnotes at page bottom
\usepackage[bookmarksnumbered=true,
            bookmarksopen=true,
            colorlinks=true,
            linkcolor=blue,
            anchorcolor=blue,
            citecolor=blue
           ]{hyperref}

\makeindex             % used for the subject index

\begin{document}
\begin{CJK}{UTF8}{gkai}
\frontmatter%%%%%%%%%%%%%%%%%%%%%%%%%%%%%%%%%%%%%%%%%%%%%%%%%%%%%%
\include{titlepage}
\include{preface}
\tableofcontents
\include{notation}
\mainmatter%%%%%%%%%%%%%%%%%%%%%%%%%%%%%%%%%%%%%%%%%%%%%%%%%%%%%%%
该神仙,知何途;混样近,连续组\\
问问这个神仙,知不知道哪条路径是对的,妖怪(混子)走过来,
连连阻止前进
\begin{enumerate}
\item 该:概率论
\item 神:神经网络
\item 仙:线性回归和线性分类,广义线性模型
\item 知:支持向量机
\item 何:核函数
\item 途:图模型
\item 混:混合模型
\item 样:抽样
\item 近:近似推断
\item 连:连续隐变量
\item 续:序列化数据
\item 组:组合模型
\end{enumerate}
\include{introduction}
\chapter{Probability Distribution}
\label{probability_distribution}

\section{高斯分布}
\begin{equation}
N(x|\mu, \Sigma), \Lambda \equiv \Sigma^{-1}
\end{equation}

\begin{equation}
\begin{aligned}
x = \begin{pmatrix}
 x_a\\ 
 x_b
  \end{pmatrix},
\mu = 
\begin{pmatrix}
 \mu_a \\ 
 \mu_b
  \end{pmatrix}
\end{aligned}
\end{equation}

\begin{equation}
\begin{aligned}
\Sigma = \begin{pmatrix}
\Sigma_{aa}&\Sigma_{ab}\\
\Sigma_{ba}&\Sigma_{bb}
\end{pmatrix},
\Lambda = 
\begin{pmatrix}
\Lambda_{aa}& \Lambda_{ab}\\
\Lambda_{ba}& \Lambda_{bb}
\end{pmatrix}
\end{aligned}
\end{equation}

\subsection{条件分布}
\begin{equation}
\begin{aligned}
p(x_a|x_b) = N(x_a|\mu_{a|b}, \Lambda_{aa}^{-1})
\\
\mu_{a|b} = \mu_a - \Lambda_{aa}^{-1}\Lambda_{ab}(x_b - \mu_b)
\end{aligned}
\end{equation}

\subsection{边缘分布}
\begin{equation}
\begin{aligned}
p(x_a) = N(x_a|\mu_a, \Sigma_{aa})
\end{aligned}
\end{equation}

\section{贝叶斯推理}
\begin{equation}
p(x) = N(x|\mu, \Lambda^{-1})
\end{equation}

\begin{equation}
p(y|x) = N(y|Ax + b, L^{-1})
\end{equation}

\begin{equation}
p(y) = N(y|A\mu + b, L^{-1} + A\Lambda^{-1}A^T)
\end{equation}

\begin{equation}
p(x|y) = N(x|\Sigma\{A^TL(y-b) + \Lambda\mu\}, \Sigma)
\end{equation}

\begin{equation}
\Sigma = (\Lambda + A^TLA)^{-1}
\end{equation}

\subsection{伽玛分布}
精度的符号问题,$\alpha$是一种常用记号,在线性回归里面
经常使用$\alpha,\beta$,使用$\lambda$也很容易理解,在线性代数里面
$\lambda$经常用表示特征值,协方差矩阵用SAS分解以后,就是一个对角
阵,每个元素就是特征值,这也就很容易理解为什么精度矩阵用$\Lambda$了,
因为它本就是特征值的矩阵。
伽玛分布常作为高斯分布的共轭出现。

\begin{equation}
Ga(\lambda|a, b) = \frac{b^a}{\Gamma(a)}\lambda^{a-1}e^{-b\lambda}
= z \lambda^{a-1}e^{-b\lambda}
\end{equation}

\begin{equation}
p(x|\lambda) = \prod_{n=1}^NN(x_n|\mu, \lambda^{-1})
= z \lambda^{\frac{N}{2}}e^{-\frac{\lambda}{2}\sum_{n=1}^N(x_n-\mu)^2}
= z \lambda^ke^{-l \lambda}
\end{equation}

\subsection{高斯伽玛分布}
\begin{equation}
p(\mu, \lambda) = p(\mu|\lambda)p(\lambda)
= N(\mu|\mu_0, (\beta \lambda)^{-1})Ga(\lambda|a, b)
\end{equation}

\subsection{指数族}

\begin{enumerate}
\item 指数族
\begin{equation}
p(x|\eta) = h(x)g(\eta)exp\{\eta^Tu(x)\}
\end{equation}
\end{enumerate}

\subsection{极大似然估计和充分统计}

\begin{enumerate}
\item 似然函数
\begin{equation}
p(X|\eta) = (\prod_{n=1}^Nh(x_n))g(\eta)^N \exp \{
\eta^T \sum_{n=1}^N u(x_n)
\}
\end{equation}
\item 充分统计
\begin{equation}
-\nabla\ln(\eta_{ML}) = \frac{1}{N}\sum_{n=1}^N{u(x_n)}
\end{equation}
\end{enumerate}

\subsection{共轭先验}
\begin{equation}
p(\eta|x, v) = f(x, v)g(\eta)^v\exp\{v\eta^Tx\}
\end{equation}


\subsection{问题}
\begin{enumerate}
\item 如何根据2.228的定义求出$\eta_{ML}$?\\
根据公式2.203, 2.217以及2.223可以知道,$g(\eta)$与均值与方差的关系,
因此只要我们知道了均值
与方差就可以知道$\eta_{ML}$。$\ln g(\eta)$称作配分函数,
配分函数的一个很好性质就是它的导数就$E(u(x))$,二次导
数是$var(u(x))$,因此由2.228可以得到$\eta$。
\end{enumerate}

\chapter{Linear models for regression}
\label{linear_models_regression}
\section{贝叶斯线性回归}
\subsection{参数分布}
公式符号与第\ref{probability_distribution}、
\ref{sparse_kernel_machines}章保持一致, 见Notation.

\begin{enumerate}
\item 参数后验分布一般形式
\begin{equation}
p(w|t) = N(w|\mu_N, \Sigma_N)
\end{equation}

\begin{equation}
\mu_N = \Sigma_N(\Sigma_0^{-1}m_0 + \beta \Phi^Tt)
\end{equation}

\begin{equation}
\Sigma_N^{-1} = \Sigma_0^{-1} + \beta\Phi^T\Phi
\end{equation}

\item 参数后验分布常用形式
\begin{equation}
p(w|t) = N(w|0, \alpha^{-1}I)
\end{equation}

\begin{equation}
\mu_N = \beta \Sigma_N\Phi^Tt 
\end{equation}

\begin{equation}
\Sigma_N^{-1} = \alpha I + \beta \Phi^T\Phi
\end{equation}

\end{enumerate}
参数分布验精度矩阵等于先验精度矩阵加数据精度矩阵

\section{经验贝叶斯}
同名经验贝叶斯,第二类极大似然估计、广义极大似然估计、模型近似

\begin{equation}
p(t|D) = \int\int\int p(t|w, \beta)p(w|\alpha, \beta)
p(\alpha,\beta|D)dwd\alpha d\beta
\end{equation}

近似预测

\begin{equation}
p(t|D) \simeq p(t|\hat{\alpha}, \hat{\beta}) = \int p(t|w, \hat \beta)
p(w|\hat \alpha, \hat \beta)dw
\end{equation}

\begin{equation}
\frac{dm_N}{d\alpha} = A^{-1}m_N
\end{equation}

\begin{equation}
\frac{dm_N^T}{d\alpha} = m_N^TA^{-1}
\end{equation}

\begin{equation}
\begin{aligned}
E(m_N) = \frac{\beta}{2}
(t^Tt - 2t^T\Phi m_N + m_N^T\Phi^T\Phi m_N) + \frac{\alpha}{2}m_N^Tm_N
= \frac{\beta}{2}t^Tt - \beta t^T\Phi m_N + \frac{1}{2}m_N^Tm_N \\
= \frac{\beta}{2}t^Tt - m_N^TAm_N + \frac{1}{2}m_N^TAm_N 
= \frac{\beta}{2}t^Tt - \frac{1}{2}m_N^TAm_N
\end{aligned}
\end{equation}

\begin{equation}
\begin{aligned}
dE(m_N) = -\frac{1}{2}dm_N^TAm_N = -\frac{1}{2}\beta^2dt^T\Phi A^{-1}AA^{-1}
\Phi^Tt = \frac{1}{2}\beta^2t\Phi A^{-1}A^{-1}\Phi^Tt
= \frac{1}{2}m_N^Tm_N
\end{aligned}
\end{equation}

\subsection{问题}
\begin{enumerate}
\item 极大似然使用条件
\end{enumerate}

\section{联系}
\begin{enumerate}
\item 经验贝叶斯、模型估计,贝叶斯线性回归,贝叶斯线性分类
、RVM回归,RVM分类,高斯过程
\end{enumerate}

\chapter{Linear Models for Classification}
\section{输入变量是离散型}
\subsection{联合分布}
\begin{enumerate}
\item y服从伯努利分布(二分类)
\begin{equation}
\begin{aligned}
p(x, y) = p(y)p(x|y) = \pi^y(1-\pi)^{1-y}
p(x|\mu_0)^{1-y}p(x|\mu_1)^y
= [(1-\pi)p(x|\mu_0)]^{1-y}[\pi p(x|\mu_1)]^y
\end{aligned}
\end{equation}
\begin{enumerate}
\item x服从伯努利分布
\begin{equation}
p(x, y) =[(1-\pi)\mu_0^x(1-\mu_0)^{1-x}]^{1-y}
[\pi\mu_1^x(1-\mu_1)^{1-x}]^y
\end{equation}
\item x服从离散分布
\begin{equation}
p(x, y) =[(1-\pi)\prod_{k=1}^K\mu_{0k}^{x_k}]^{1-y}
[\pi \prod_{k=1}^K\mu_{1k}^{x_k}]^y
\end{equation}
\end{enumerate}

\item y服从离散分布(多分类)
\begin{equation}
p(x, y) = p(y)p(x|y) = \prod_{k=1}^K\pi_k^{y_k}
\prod_{k=1}^Kp(x|\theta_k)^{y_k}
=\prod_{k=1}^K[\pi_kp(x|\theta_k)]^{y_k}
\end{equation}
\begin{enumerate}
\item x服从伯努利分布
\begin{equation}
\begin{aligned}
p(x, y) =\prod_{k=1}^K[\pi_k\mu_k^x(1-\mu_k)^{1-x}]^{y_k}
\end{aligned}
\end{equation}

\item x服从离散分布
\begin{equation}
\begin{aligned}
p(x, y) =\prod_{c=1}^C[\pi_k\prod_{k=1}^K\mu_{ck}^{x_k}]^{y_c}
\end{aligned}
\end{equation}

\end{enumerate}
\end{enumerate}

\subsection{极大似然估计}
\begin{enumerate}
\item $y\sim B(y|\pi), x\sim B(x|\mu)$
\begin{equation}
logp(D) = \sum_{n=1}^N
\{(1-y)[log(1-\pi) + xlog\mu_0 + (1-x)log(1-\mu_0)]
+ y[log\pi + xlog\mu_1 + (1-x)log(1-\mu_1)]\}
\end{equation}

令
\begin{equation}
\frac{\partial logp(D)}{\partial \pi}
= \frac{1}{\pi}\sum_{n=1}^Ny - \frac{1}{1-\pi}\sum_{n=1}^N(1-y)
=0
\end{equation}
得
\begin{equation}
\pi = \frac{\sum y}{N}
\end{equation}

令
\begin{equation}
\frac{\partial logp(D)}{\partial \mu_1}
= \frac{1}{\mu_1}\sum_{n=1}^Nyx - \frac{1}{1-\mu_1}\sum_{n=1}^Ny(1-x)
=0
\end{equation}
得
\begin{equation}
\mu_1 = \frac{\sum xy}{\sum y}
\end{equation}

令
\begin{equation}
\frac{\partial logp(D)}{\partial \mu_0}
= \frac{1}{\mu_0}\sum_{n=1}^N(1-y)x - \frac{1}{1-\mu_0}\sum_{n=1}^N(1-y)(1-x)
=0
\end{equation}
得
\begin{equation}
\mu_0 = \frac{\sum x(1-y)}{\sum (1-y)}
\end{equation}

\item $y\sim Cat(y|\pi), x\sim Cat(x|\mu)$
\begin{equation}
\begin{aligned}
logp(D) =\sum_{n=1}^N\sum_{c=1}^Cy_{nc}
[log\pi_k + \sum_{k=1}^Kx_{nk}log\mu_{ck}]
\end{aligned}
\end{equation}
另外
\begin{equation}
\sum_{k=1}^K\mu_{ck} = 1
\end{equation}
最大化下面的式子
\begin{equation}
\begin{aligned}
\sum_{n=1}^N\sum_{c=1}^Cy_{nc}[log\pi_k + \sum_{k=1}^Kx_{nk}log\mu_{ck}]
+ \lambda (\sum_{k=1}^K\mu_{ck} - 1)
\end{aligned}
\end{equation}
得
\begin{equation}
\mu_{ck} = \frac{\sum_{n=1}^N y_{nc}x_{nk}}{\sum_{n=1}^N\sum_{k=1}^K
y_{nc}x_{nk}}
= \frac{N_{ck}} {N_c}
\end{equation}


\end{enumerate}

\section{朴素贝叶斯}
似然函数
\begin{equation}
p(D|\pi, \theta) = \prod_{n=1}^N\{
(\prod_{c=1}^C\pi_c^{y_c^n})
(\prod_{m=1}^M\prod_{c=1}^Cp(x_m^n|\theta_{mc})^{y_c^n}
\}
\end{equation}

对数似然函数
\begin{equation}
logp(D|\pi, \theta) = \sum_{c=1}^CN_clog\pi_c
+ \sum_{m=1}^M\sum_{c=1}^C\sum_{n=1}^Ny_c^nlogp(x_m^n|\theta_{mc})
\end{equation}
用极大似然估计, $\pi_c = \frac{N_c}{N}$

以上是mlapp中给出的公式,太复杂,不够简练,简练的公式容易让人记忆,
容易提炼出本质。下面给出一种紧凑的格式。

似然函数
\begin{equation}
p(D|\pi, \theta) = \prod_{c,n}
[\pi_c\prod_mp(x_m^n|\theta_{mc})]^{y_c^n}
\end{equation}

对数似然函数
\begin{equation}
logp(D|\pi, \theta) = \sum_{c, n}y_c^n
[log\pi_c + \sum_mlogp(x_m^n|\theta_{mc})]
\end{equation}
\subsection{贝努力分布}

\begin{equation}
p(x_m^n|\theta_{mc}) = p(x_m^n|\mu_{mc})
= \mu_{mc}^{x_m^n}(1 - \mu_{mc})^{1 - x_m^n}
\end{equation}

对数似然函数

\begin{equation}
\begin{aligned}
logp(D|\pi, \theta) = \sum_{c,n}
y_c^n\{log\mu_c + \sum_m[x_m^nlog\mu_{mc} + (1 - x_m^n)
log(1 - \mu_{mc})]\}
\\= \sum_c\{
N_clog\pi_c + \sum_m[N_{mc}log\mu_{mc} 
+ (N_c - N_{mc})log(1-\mu_{mc})]
\}
\end{aligned}
\end{equation}

用极大似然估计

\begin{equation}
\frac{\partial logp(D|\pi, \theta)} {\partial \mu_{mc}}
= \mu^{-1}_{mc}N_{mc} - (1 - \mu_{mc})^{-1}(N_c - N_{mc})
\end{equation}

令
\begin{equation}
\frac{\partial logp(D|\pi, \theta)} {\partial \mu_{mc}}
= 0
\end{equation}
得
$\mu_{mc} = \frac{N_{mc}}{N_c}$

\subsection{离散分布}
\begin{equation}
Cat(x_m^n|\mu_{mc}) = \prod_{k = 1}^K\mu_{mck}^{x_{mk}^n}
\end{equation}

离散分布似然函数
\begin{equation}
p(D|\pi, \theta) = \prod_{n=1}^N\{
(\prod_{c=1}^C\pi_c^{y_c^n})
(\prod_{m=1}^M\prod_{c=1}^C
\{\prod_{k = 1}^K\mu_{mck}^{x_{mk}^n}
\}^{y_{c}^n}
\}
\end{equation}
写成简洁格式如下:
\begin{equation}
p(D|\pi, \theta) = \prod_{c, n}
(\pi_c\prod_{m, k}\mu_{mck}^{x_{mk}^n})^{y_c^n}
\end{equation}
相应的对数似然函数
\begin{equation}
\begin{aligned}
logp(D|\pi, \theta) = \sum_{c, n}y_c^n[log\pi_c + \sum_{m, k}x_{mk}^nlog\mu_{mck}]
\\= \sum_c[N_clog\pi_c + \sum_{m, k}log\mu_{mck}\sum_ny_c^nx_{mk}^n]
\\=\sum_c[N_clog\pi_c + \sum_{m,k}N_{mck}log\mu_{mck}]
\end{aligned}
\end{equation}
注意到
\begin{equation}
\sum_{m,k} \mu_{mck} = 1
\end{equation}
因此,可以通过最大化下面的式子求解:
\begin{equation}
\sum_{m,c,k}N_{mck}log\mu_{mck} +\lambda(\sum_{m,k}\mu_{mck} -1)
\end{equation}
求导后,可得:
\begin{equation}
\mu_{mck} = -\frac{N_{mck}}{\lambda}
\end{equation}
根据条件概率定义:$\sum_{m,k}\mu_{mck} = 1$, 而不是$\sum_{m,c,k}\mu_{mck} = 1$, 
得$\mu_{mck} = \frac{N_{mck}} {N_c}$
从以上可以看出,朴素贝叶斯提供了一个框架,针对不同特征的具体分布,
代入基本公式可以用极大似然求出参数。

\section{判别式模型}
\subsection{概率回归}
\begin{enumerate}
\item 激活函数
\begin{equation}
f(a) = E(a) = \int_{-\infty}^ap(\theta)d\theta
\end{equation}
\item 概率函数
\begin{equation}
\Phi(a) = \int_{-\infty}^a
f(a) = E(a) = \int_{-\infty}^aN(\theta|0, 1)d\theta)
\end{equation}
\end{enumerate}
\subsection{标准链接函数}
\section{问题}
\begin{enumerate}
\item 公式4.146
\item kernel logistiction regression
\end{enumerate}

\section{Bayesian LR}
\subsection{Laplace 近似}



\include{neural_networks}
\chapter{Kernel Methods}
\section{问题}
\begin{enumerate}
\item 什么是核?
\item 什么是高斯过程?
\item 为什么使用高斯过程?
\item P292 固定核?
\item 对偶形式的表达,出现了参数个数的变化,由$|w|$变为
$|a|$,是否会过拟合,能带来哪些好处,特征多的影响,特征少的
影响。例如文档分类,特征多,训练集少。
\end{enumerate}
\section{核}
\begin{enumerate}
\item 高斯核
\begin{equation}
k(x_m, x_n) = exp(-||x_m - x_n||^2/2\sigma^2)
\end{equation}
\item 指数核
\begin{equation}
k(x_m, x_n) = exp(-\theta|x_m - x_n|)
\end{equation}
\end{enumerate}

\section{对偶表示}
\subsection{线性回归}
\begin{equation}
y(x) = w^T\phi(x)
\end{equation}
\subsection{对偶表示}
\begin{equation}
y(x) = a^T\Phi\phi(x) = a^Tk(x)
\end{equation}
\subsection{二者之间的关系}
\begin{equation}
J(a) = \frac{1}{2}\sum_{n=1}^N(k(x_n)^Ta - t_n)^2
= \frac{1}{2}||\Phi\Phi^Ta - t||^2
\end{equation}

\begin{equation}
J(w) = \frac{1}{2}\sum_{n=1}^N(\phi(x_n)^Tw - t_n)^2
= \frac{1}{2}||\Phi w - t||^2
\end{equation}

\subsection{协方差矩阵}

\begin{equation}
cov(y) = E(yy^T) = E(\Phi ww^T\Phi^T)
= \Phi E(ww^T) \Phi^T
= \Phi cov(w)\Phi^T
\end{equation}

\begin{equation}
C_{N+1} = \begin{pmatrix}
C_N & k\\
k^T & c
\end{pmatrix}
\end{equation}

对偶形式的优点是,不用显示的写出特征空间向量,只关心
$k(x_m, x_n)$, 核函数完成了由特征空间到核空间的转化。
最重要的是,可以使用不同的核函数进行替换。

\section{高斯过程}
\begin{enumerate}
\item 高斯过程是定义在y上联合分布服从高斯分布的概率分布,
简单说就是高斯过程就是
定义在y上的高斯分布。高斯过程可以由完全由二次统计决定,即均值和方差。
根据高斯分布的性质,高斯分布的边缘分布和条件分布都是高斯分布。
y的取值范围是R,因此是一个无穷维。在实际应用中,只考虑训练集大小
的维度。
回归问题的目标是$p(y_{N+1}) = N(y_{N+1}|\mu_{N+1}, \sigma_{N+1})$。
因此用高斯过程解决回归问题的关键是如何求出均值和方差。
高斯过程中的协方差$E[y(x_n)y(x_m)]$可以用核的方式给出来。

定一个实矩阵 A,矩阵 $A^TA$ 是 A 的列向量的格拉姆矩阵,
而矩阵$AA^T$ 是 A 的行向量的格拉姆矩阵。
\item 高斯过程与核函数\\
高斯过程协方差矩阵自然的与核函数联系在一起。

\end{enumerate}
\section{Summary}
\begin{enumerate}
\item 高斯过程与线性回归之间的对偶
\item 高斯过程与隐变量,对于输出y,y和y之间
实际上是有关系的,通过指定条件w(相当于一个隐变量),
从而使得y只依赖于w和x,与产生式模型的比较。
\item 高斯过程与线性模型\\
线性模型用w表示了x和y之间的关系,并且通过w之间的
协方差表示y之间的协方差。高斯过程抛弃了w,直接表示
y之间的关系,这种关系通过核表示出来。高斯过程用核$k(x_m,x_n)$
取代了线性模型中的w。
\item 回归与分类,回归问题+一个激活函数,转变为分类问题
\end{enumerate}



\chapter{Sparse kernel machines}
\section{问题}
\begin{enumerate}
\item 当数据量很大的时候,计算所有核函数对很复杂,
如何只利用局部核解决问题? 即如何通过少量核求解问题?
\item 什么是无穷维?
\item 对偶表示的好处?
\end{enumerate}

\chapter{Graphical models}
\section{简介}
图模型提供了一个工具,描述问题,模拟问题,推理问题\cite{longxing2012machinelearning}.
三个基本问题,
\begin{enumerate}
\item 有了问题,如何模拟,怎么描述,把自己的先验知识放到问题里面; 
\item 推理,有了模型, 可以去回答什么问题;
\item 学习,参数不了解,结构化的东西不了解,通知数据学习参数和结构;
\end{enumerate}
用紧凑的方式表示变量之间的关系。
\begin{enumerate}
\item 什么样的图模型更适合一个问题?
\item 对定一个图模型,求P(x|y)
\item 如何得到图模型的结构和参数
\end{enumerate}

\section{条件独立性}
给定一个联合分布,我们关心哪些变量之间是条件独立的。或者在给定条件下判断两个
变量是否独立。

\begin{equation}
A\bot B | C
\end{equation}
\section{贝叶斯网}
\subsection{有向图过滤器}
\begin{enumerate}
\item 有向图有什么用?\\
更方便的表达独立性,一个有向图对应着一个联合分布的一个
形式,我们关心的是该有向图表达了什么样的独立性,
即在给定什么样的条件下两个变量是独立的。
独立性表达能力是一个图的本质属性。如果两个图的独立性表达能力
相同,可以认为是等价的。
一个有向图最直观的分解方式就是父节点分解法。
条件独立性的重要性在于把本不独立的两个事件看作是独立的,例如朴素
贝叶斯。
如果需要的条件越少,那么可以认为独立性越强,相反信赖性越强。
\item 有向分离理论,(过滤器原理)\\
给定一个有向图,可以直接写出一个分解形式,每个因式是一个条件分布;
同样,按照有向分离原则,也可以得到一种条件独立性;
有向分离理论告诉我们这两种形式的条件独立性是等价的。
这种等性的意义在于用一个可以直观容易表达的公式表达了一种复杂的需求。
因式分解是数学表达,条件独立是需求。\\
把一个有向图看作一个过滤器,第一种分布是让满足图分解的分布通过,
第二种是让满足有向分立的分布通过,有向分离理论告诉我们这两种分布集合
等价。即一个图的分解分布和有向分立分布在独立性上等价的。\\
两个极限,全连通图和全分解分布,全连通图可以让任意一个分布通过?全分解
分布可以通过任意图。哪果一个分布可以被有向图表达,那么它一定可以通过
全连通图。
因为有些分布有向图是无法表达的,是否意味着有些分布
不能通过全连通图。
\end{enumerate}

有向分离是判断贝叶斯网中两个节点是否条件独立的方法。
图的表达能力,越连通表达能力越强,越独立表达能力越弱。
全联全概率或者全连通图表达能力最强,可以表达一切分布,
因此可以让所有分布都通过。对于一个完全分解的图,表达能力最弱。

\subsection{有向分离}
\begin{enumerate}
\item 高斯分布
\begin{equation}
p(D|\mu) = \prod_{n=1}^Np(x_n|\mu)
\end{equation}

\begin{equation}
p(D) \neq \prod_{n=1}^Np(x_n)
\end{equation}

\item 朴素贝叶斯
\begin{equation}
p(x|z) = \prod_{m=1}^M(x_m|z)
\end{equation}
\end{enumerate}


\section{概念}
\begin{enumerate}
\item 条件独立性
\item 有向分离
\end{enumerate}

\chapter{Mixture models and EM}
\section{问题}
\begin{enumerate}
\item 经验贝叶斯和EM算法的关系
\end{enumerate}

\chapter{Approximate inference}
\section{变分推理}
\section{可分解分布}
\begin{equation}
q(Z) = \prod_{m=1}^Mq_m(Z_m)
\end{equation}


\include{sampling_methods}
\chapter{Continuous latent variables}
\section{问题}
\begin{enumerate}
\item PPCA与产生式模型
\item 从自由度理解PCA
\item PCA、PPCA和FA\\
PCA强调数据空间到隐空间,PPCA和FA强调隐空间到数据空间
\end{enumerate}

\include{sequential_data}
\include{combining_models}
\chapter{Question}
\section{问题}

\begin{enumerate}
\item PCA降维与隐变量降维之间的关系?\\
\item 矩阵稀疏什么时候好,什么时候坏?\\
一般稀疏的矩阵容易处理,网络社团算法一般都是在
稀疏矩阵中做的。对于推荐问题,稀疏以后就面临过拟合,推荐不准确的困难。
\item 似然函数的写法?\\
回归问题、产生式模型、判别式模型
\begin{equation}
p(D|w) = \prod^N_{n=1} p(y_n|w^Tx_n)
\end{equation}
分类问题
\begin{equation}
p(D|\theta) = \prod_{n=1}^N p(x_n, y_n|\theta)
= \prod_{n=1}^Np(x_n|\theta)p(y_n)
\end{equation}
\item 高斯过程图模型?
\item EM算法,广义EM算法,非参方法EM算法?
\item 是否可以把分类问题的标签去掉变成聚类问题?
\end{enumerate}

\section{EM算法}
\begin{enumerate}
\item EM算法是否可以用于没有隐变量的情况?
\item 可不可以把隐变量当成参数使用梯度下降算法?
\item FA可以使用梯度下降算法吗?
\end{enumerate}
\section{Idea}
\begin{enumerate}
\item 熟悉的内容讲公式
\item 不熟悉的内容讲直观
\item 熟悉以后再讲公式
\item 组成一个加强组 
\end{enumerate}

\section{目标}
\begin{enumerate}
\item 发高质量论文
\item 至少精读一本书(PRML、MLAPP)
\item 集中阅读文献(略读)
\end{enumerate}




\backmatter%%%%%%%%%%%%%%%%%%%%%%%%%%%%%%%%%%%%%%%%%%%%%%%%%%%%%%%
\appendix
%\include{appendix}
\include{glossary}
\printindex
\clearpage
\end{CJK}

\bibliographystyle{plain}
\bibliography{mybib}

\end{document}
